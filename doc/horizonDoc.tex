\documentclass[12pt, a4paper]{article}


%http://pgfplots.sourceforge.net/pgfplots.pdf

\usepackage{pdfsync}
\usepackage{hyperref}
\usepackage{amsfonts }
%\usepackage[pdftex]{graphicx}
\usepackage{float}
\usepackage{authblk}
\usepackage{ifthen}
\usepackage{subfigure}

\usepackage{pgfplots} 
\usepackage{filecontents}

\usepackage[toc,page]{appendix}

%%%% sets paper dimensions as A4
\setlength{\paperwidth}{210mm} 
\setlength{\paperheight}{297mm}

%%%%%% Text dimensions

\oddsidemargin 0 mm
\evensidemargin 0 mm
\textwidth 160 mm
\topmargin -3 mm
\textheight 240 mm

\newcommand{\todo}{{\bf TODO:}}

\newcommand\be{\begin{eqnarray}}    
\newcommand\ee{\end{eqnarray}}	
\newcommand\ba{\begin{array}}	   
\newcommand\ea{\end{array}}
\newcommand\eeq{\end{equation}}	 	
\newcommand\beq{\begin{equation}}



\newcommand{\nn}{\nonumber}

\newcommand{\KK}{{\cal{K}}}

\newcommand{\BS}{{\mathrm{{\cal{BS}}}}}


\newcommand{\SSS}{{\cal{S}}}

\newcommand{\BBB}{{\cal{B}}}

\newcommand{\WWW}{{\cal{W}}}


\newcommand{\DEF} {{\bf{Definition: }}}
\newcommand{\PROP} {{\bf{Proposition: }}}
\newcommand{\PROOF} {{\emph{Proof: \\ \\}}}



\newcommand{\AAA} {\mathcal{A} }
\newcommand{\LL} {\mathcal{L} }
\newcommand{\FF} {\mathcal{F} }
\newcommand{\GG} {\mathcal{G} }
\newcommand{\DPO} {dP(\omega) }
\newcommand{\BOREL} {{\mathcal{B}(\Re)} }
\newcommand{\FLT} {\underline{\mathcal{F}}} 
\newcommand{\TS} {\tilde{S}} 
\newcommand{\TX} {\tilde{X}} 
\newcommand{\OW} {\overrightarrow{w}} 
\newcommand{\OS} {\overrightarrow{\sigma}} 
\newcommand{\OV} {\overrightarrow{v}} 
\newcommand{\OZ} {\overrightarrow{z}} 
\newcommand{\NN} {\mathcal{N}} 

\newcommand{\lp} {\left( }
\newcommand{\rp} {\right)}
\newcommand{\lpq} {\left[}
\newcommand{\rpq} {\right]}
\newcommand{\lpg} {\left\{ }
\newcommand{\rpg} {\right\} }



\title{\textbf{Horizon Documentation}}


\author{Horizons}
\affil{Horizon}

\date{\today}


\begin{document}


\maketitle

\section{Notation}
Unless explicitly stated, the following symbols will be used with the meaning defined here below:
\begin{enumerate}
\item $N$: standard cumulative normal
\item: $n(z)=\frac{e^{-\frac{z^2}{2}}}{\sqrt{2\pi}}$: standard normal probability density
\end{enumerate}

\section{Normal Black-Scholes Model}

{\bf{A. Cesarini 20140601-NBS}} \\
Let us consider an asset having the following dynamics for the evolution from $t$ to a certain maturity $T>t$:
\be
S(T) = F+\sigma z
\ee
where $z$ is a standard normal and $F$ is called $T$-maturity asset forward.
Let us compute the undiscounted price of a vanilla option on $S(T)$:
\be
\Pi[\phi] = E\left[ \left( \phi (S(T)-K) \right)^+\right]= E\left[ \left( \phi (\sigma z - (K-F)) \right)^+\right]
\ee
We have
\be
\Pi[\phi] = \int_{z^*}^{\phi \infty} n(z)  (\sigma z - (K-F)) dz 
\ee  
with 
\be
z^* = \frac{K-F}{\sigma}
\ee
which gives
\be
\Pi[\phi] = \phi (F-K) N(-\phi z^*) + \sigma n(z^*)
\ee


\section{Girsanov's Theorem}

\PROP (From \cite{Bjork}) Let $W^P$ be a $d$-dimensional standard (i.e. independent components) $P$-Wiener process on $(\Omega, \FF, P, \FLT)$ and let $\varphi$ be any $d$-dimensional adapted column vector process. Choose a fixed $T>0$ and define the (scalar) process $L$ on $[0,T]$ by
\[
dL_t=\varphi_t^* L_t dW_t^P; L_0=1
\]
that is
\[
L_t=\exp\left[\int_0^t \varphi_s^* dW_s^P -\frac{1}{2}\int_0^t \| \varphi_s\|^2 ds\right]
\]
Assume that 
\[
E^P[L_T]=1
\]
and define the new probability measure $Q$ on $\FF_T$ by
\[
L_T=\frac{dQ}{dP} 
\]
Then 
\be
dW_t^P&=&\varphi_t dt + dW_t^Q \label{girseq1}\\
W_t^Q&=&W_t^P -\int_0^t \varphi_s ds\label{girseq2}
\ee
where $W^Q$ is a $Q$-Wiener process. \\ \\
\PROOF
\todo

\section{Changing measures between equivalent martingale measures}
For every non dividend paying tradable asset $H$, measurable on $\FF_{\tau}$, and considering two numerairs $N(\tau)$, $M(\tau)$, we know that the following holds ($t\leq \tau$):
\[
H(t)=N(t) E^N\left[\frac{H(\tau)}{N(\tau)} | \FF_t \right]=M(t) E^M\left[\frac{H(\tau)}{M(\tau)} | \FF_t \right]
\]
which, defining $G(\tau)=\frac{H(\tau)}{N(\tau)}$, yields
\[
E^N\left[G(\tau)| \FF_t \right]= E^M\left[ G(\tau) \frac{M(t)}{M(\tau)} \frac{N(\tau)}{N(t)}| \FF_t \right]= E^M\left[G(\tau) \frac{dQ^N}{dQ^M}(\tau) | \FF_t \right]
\]
which gives
\be 
\frac{dQ^N}{dQ^M}(\tau) =  \frac{M(t)}{M(\tau)} \frac{N(\tau)}{N(t)}
\label{radniknumeraires}
\ee


\section{Changing measures between $T$-fwd martingale measures \label{t_fwd_measure_changes}}
{\bf{A. Cesarini 20140823-CONVADJ}} \\
Let $t_0<t_1<\cdots < t_{N}$ be a time schedule and referring to (\ref{radniknumeraires}) we assume
\be 
M(t) = P(t, t_k) \\
N(t) = P(t, t_h)
\ee
with $t\le \tau \le \min(t_k, t_h)$ and $h,k\ge 0$. By $P(\tau, T)$ we denote the risk free zero coupon bond price as observed from $\tau$ for maturity $T$, delivering $1$ unit of currency at $T$. We then get
\be 
\frac{dQ^N}{dQ^M}(\tau) =  \frac{dQ^h}{dQ^k}(\tau) =  \frac{P(t, t_k)}{P(\tau, t_k)} \frac{P(\tau, t_h)}{P(t, t_h)}=L_{t, h/k}(\tau)
\label{radniknumeraires_tfwdhk}
\ee
where $L_{t, h/k}(t)=1$ and 
\be 
E^h\left[G(\tau)| \FF_t \right] = E^k\left[G(\tau) L_{t, h/k}(\tau) | \FF_t \right]
\ee
We now write
\be 
P(\tau, h/k) \equiv \frac{P(\tau, t_h)}{P(\tau, t_k)}= \left\{\Pi_{i=m_{h,k}+1}^{M_{h,k}}\left[ 1 + F_i(\tau)\tau_i \right]\right\}^{s(h,k)} \label{phk}\\
m_{h,k} = \min(h, k)\\
M_{h,k} = \max(h, k) \\
s(h,k) = 1 \mbox{ if $h\le k$}, -1 \mbox{ if $h> k$}
\ee
and 
\be
F_i(\tau)\equiv \left(\frac{P(\tau, t_{i-1})}{P(\tau, t_i)}-1\right)\frac{1}{\tau_i}
\ee
We now compute the following Ito differential in the $t_k$-forward measure
\be 
d P(\tau, h/k) = \sum_{l=m_{h,k}+1}^{M_{h,k}} d F_l(\tau) \frac{\partial}{\partial F_l(\tau)}  P(\tau, h/k) =\\
=\sum_{l=m_{h,k}+1}^{M_{h,k}}\tau_l  s(h,k)  d F_l(\tau)\left[ 1 + F_l(\tau)\tau_l \right]^{s(h,k)-1}  \left\{\Pi_{i=m_{h,k}+1, i\neq l}^{M_{h,k}}\left[ 1 + F_i(\tau)\tau_i \right]\right\}^{s(h,k)}= \\
= s(h,k) P(\tau, h/k) \sum_{l=m_{h,k}+1}^{M_{h,k}} \frac{\tau_l  d F_l(\tau)}{1 + F_l(\tau)\tau_l } 
\label{dphk}
\ee
From the last equation and (\ref{radniknumeraires_tfwdhk}) we get (always in the $t_k$-forward measure)
\be 
d L_{t, h/k}(\tau) = L_{t, h/k}(\tau)  s(h,k) \sum_{l=m_{h,k}+1}^{M_{h,k}} \frac{\tau_l  d F_l(\tau)}{1 + F_l(\tau)\tau_l } 
\ee
We now use this last equation, together with the Girsanov's theorem (\ref{girseq1}) to get the following result.
Suppose to consider a $1$-dimensional Wiener process $Z^{(k)}(\tau)$ where the $(k)$-apex is used to stress that it is a Wiener process in the $t_k$-fwd measure. We can then conclude that $Z^{(h)}(\tau)$ as defined in the following is a Wiener process in the $t_h$-fwd measure:
\be 
dZ^{(h)}(\tau) = dZ^{(k)}(\tau)- s(h,k) \sum_{l=m_{h,k}+1}^{M_{h,k}} \frac{\tau_l  \left<d F_l(\tau)\cdot dZ^{(k)}(\tau)\right>}{1 + F_l(\tau)\tau_l } 
\label{tfwdmeasurechangedet}
\ee

\subsection{Shifted log-normal fwd rate model and measure change between $T$-fwd martingale measures}

We now assume that
\be 
F_l(\tau) = \lambda_l + f_l(\tau)
\ee
where $\lambda_l$ is a constant and $f_l(\tau)$ is the following (martingale) log-normal process in the $t_l$-fwd measure:
\be 
\frac{d f_l(\tau)}{f_l(\tau)} = -\frac{1}{2}\sigma_l^2 d\tau + \sigma_l dW_l^{(l)}(\tau)
\ee
where we used a notation that stresses that $W_l^{(l)}$ is the Wiener process driving $f_l(\tau)$ in the $t_l$-fwd measure and we introduced an annual volatility process $\sigma_l$ (not a constant in general).
We now further observe the obvious relation 
\be 
d F_l(\tau) = d f_l(\tau)
\ee
Equation (\ref{tfwdmeasurechangedet}) then becomes
\be 
dZ^{(h)}(\tau) = dZ^{(k)}(\tau)- s(h,k) \sum_{l=m_{h,k}+1}^{M_{h,k}} \frac{\tau_l \sigma_l f_l(\tau) \left<d W_l^{(l)}(\tau)\cdot dZ^{(k)}(\tau)\right>}{1 + F_l(\tau)\tau_l } =\\
= dZ^{(k)}(\tau)- s(h,k) \sum_{l=m_{h,k}+1}^{M_{h,k}} \frac{\tau_l \sigma_l (F_l(\tau)-\lambda_l) \left<d W_l^{(l)}(\tau)\cdot dZ^{(k)}(\tau)\right>}{1 + F_l(\tau)\tau_l }
\label{shifttfwdmeasurechange}
\ee
We further notice that one could apply the 'freezing the drift approximation' in the last equation to get
\be 
dZ^{(h)}(\tau) =dZ^{(k)}(\tau)- s(h,k) \sum_{l=m_{h,k}+1}^{M_{h,k}} \frac{\tau_l \sigma_l (F_l(t)-\lambda_l) \left<d W_l^{(l)}(\tau)\cdot dZ^{(k)}(\tau)\right>}{1 + F_l(t)\tau_l }
\ee
where $t$ has been substituted in place of $\tau$ in some of the factors.

\subsection{Changing measures between $T$-fwd martingale measures of different currencies}
{\bf{A. Cesarini 20140902-CCYCONVADJ}} \\
Let us consider a tradable asset $A$ quoted in currency $\alpha$ and suppose to know its market-agreed forward value as seen from time $t$ for maturity $t_k$. For example, $A$ could be a stock with $t_k$ any future maturity or a fwd-rate ibor index with $t_k$ equal to its 'natural pay date' (typically its calendar adjusted end accrual date). 
By definition of fwd contract we can write 
\be 
F_A^{\alpha}(t,t_k)P^{\alpha}(t,t_k) = E^{\alpha}\left[A(f_k) D^{\alpha}(t,t_k)| F_t\right]= P^{\alpha}(t,t_k) E^{k, \alpha}\left[A(f_k)| F_t\right]
\ee
where $D^{\alpha}$ denotes the stochastic risk neutral discount factor in $\alpha$ currency and $E^{\alpha}$ the expectation operator in the corresponding measure. 
Furthermore $f_k\le t_k$ denotes the fixing date of $A$ associated to the forward contract delivering at $t_k$ (typically $f_k=t_k$ for stocks or $f_k=t_{k-1}$ for ibor indexes). 
We can then write
\be 
F_A^{\alpha}(t,t_k) = E^{k, \alpha}\left[A(f_k)| F_t\right]
\ee
We now consider the problem of quanto-ing the forward contract in a $\beta$-currency and at the same time of changing its payment date from $t_k$ to $t_h$, with $t_h\ge f_k$. 
We define $X$ as the value of $1$ unit of $\beta$ currency expressed in $\alpha$ currency. The quanto fwd price will be
\be 
E^{\beta}\left[A(f_k) D^{\beta}(t,t_h)| F_t\right] =P^{\beta}(t,t_h) E^{h, \beta}\left[A(f_k)| F_t\right]=F_A^{\beta}(t,t_h)P^{\beta}(t,t_h)=\phi_A(t,t_h,\beta)
\label{eqa1}
\ee
where the $t_h$-fwd quantoed in $\beta$-currency is then defined as
\be 
F_A^{\beta}(t,t_h) = E^{h, \beta}\left[A(f_k)| F_t\right]
\ee
By no arbitrage, it must also hold that
\be 
\phi_A(t,t_h,\beta)=E^{\alpha}\left[A(f_k) D^{\alpha}(t,t_h) X(t_h) | F_t\right] \frac{1}{X(t)}=\\
= E^{\alpha}\left[E^{\alpha}\left[A(f_k) D^{\alpha}(t,t_h)\frac{X(t_h)}{X(t)} | F_{f_k}\right] | F_t\right] =\\
= E^{\alpha}\left[D^{\alpha}(t,f_k) A(f_k) E^{\alpha}\left[ D^{\alpha}(f_k,t_h)\frac{X(t_h)}{X(t)} | F_{f_k}\right] | F_t\right] 
\ee
Again by no arbitrage it must hold that
\be 
E^{\alpha}\left[ D^{\alpha}(f_k,t_h)\frac{X(t_h)}{X(f_k)} | F_{f_k}\right] =E^{\beta}\left[ 1 \cdot D^{\beta}(f_k,t_h) | F_{f_k}\right] = P^{\beta}(f_k,t_h)
\ee
meaning that the contract delivering $1$ unit of $\beta$ currency at $t_h$ must have the same price as seen form $f_k$ irrespective of the measure we use to compute it.
Then we get
\be 
\phi_A(t,t_h,\beta)=E^{\alpha}\left[D^{\alpha}(t,f_k) A(f_k)  \frac{X(f_k)}{X(t)}  P^{\beta}(f_k,t_h) | F_t\right] 
\label{eqa2}
\ee
Equating (\ref{eqa1}) to (\ref{eqa2}) we obtain
\be 
P^{\beta}(t,t_h) E^{h, \beta}\left[A(f_k)| F_t\right] = E^{\alpha}\left[D^{\alpha}(t,f_k) A(f_k)  \frac{X(f_k)}{X(t)}  P^{\beta}(f_k,t_h) | F_t\right]=\\
=  E^{\alpha}\left[\frac{D^{\alpha}(t,t_k)}{P^{\alpha}(f_k,t_k)} A(f_k)  \frac{X(f_k)}{X(t)}  P^{\beta}(f_k,t_h) | F_t\right]
=E^{k, \alpha}\left[\frac{P^{\alpha}(t,t_k)}{P^{\alpha}(f_k,t_k)} A(f_k)  \frac{X(f_k)}{X(t)}  P^{\beta}(f_k,t_h) | F_t\right]
\ee
where we also applied a payoff deferring formula and changed from the risk neutral expectation $E^{\alpha}$ to the $t_k$-fwd measure expectation of the $\alpha$ currency.
Summarizing, we proved that for any asset $A$ it holds that
\be 
E^{h, \beta}\left[A(f_k)| F_t\right] = E^{k, \alpha}\left[A(f_k) \frac{L_X (f_k; \alpha, \beta, t_h, t_k)}{L_X (t; \alpha, \beta, t_h, t_k)}| F_t\right]
\ee
where
\be 
L_X (\tau; \alpha, \beta, t_h, t_k) = X(\tau)\frac{P^{\beta}(\tau,t_h)}{P^{\alpha}(\tau,t_k)}=\frac{dQ^{h, \beta}}{dQ^{k, \alpha}}(\tau)
\ee
is the Radon-Nikodyn derivative.
We also write $L$ in another illuminating form
\be 
L_X (\tau; \alpha, \beta, t_h, t_k) =\frac{dQ^{h, \beta}}{dQ^{k, \alpha}}(\tau)=
X(\tau)\frac{P^{\beta}(\tau,t_h)}{P^{\alpha}(\tau,t_h)}\frac{P^{\alpha}(\tau,t_h)}{P^{\alpha}(\tau,t_k)}
\ee
We notice that $L_X (\tau; \alpha, \beta, t_h, t_k)$ is a martingale in the $t_k$-fwd measure of the $\alpha$ currency (being a $X(\tau)P^{\beta}(\tau,t_h)$ a tradable asset).
We now define the process of the forward of the $X$ fx rate for date $t_h$ as
\be 
F_X(\tau, t_h)=X(\tau)\frac{P^{\beta}(\tau,t_h)}{P^{\alpha}(\tau,t_h)}
\ee
and finally obtain (see (\ref{phk}))
\be 
L_X (\tau; \alpha, \beta, t_h, t_k) =\frac{dQ^{h, \beta}}{dQ^{k, \alpha}}(\tau)=
F_X(\tau, t_h)\cdot\frac{P^{\alpha}(\tau,t_h)}{P^{\alpha}(\tau,t_k)}=F_X(\tau, t_h)\cdot P^{\alpha}(\tau, h/k)
\label{chengecurrencyandtfwdradnyk}
\ee
We now write (in the $t_k$-fwd measure of the $\alpha$ currency)
\be 
d F_X(\tau, t_h) = \cdots dt +\sigma_{F_X(t_h)}  F_X(\tau, t_h)  dW_X(\tau)
\ee
We now consider the following Ito differential (see (\ref{dphk})):
\be 
\frac{d L_X (\tau; \alpha, \beta, t_h, t_k)}{ L_X (\tau; \alpha, \beta, t_h, t_k)}=\cdots dt+ \sigma_{F_X(t_h)}dW_X(\tau) + \frac{1}{P^{\alpha}(\tau, h/k)}d P^{\alpha}(\tau, h/k)=\\
= \cdots dt+ \sigma_{F_X(t_h)}dW_X(\tau) +s(h,k) \sum_{l=m_{h,k}+1}^{M_{h,k}} \frac{\tau_l  d F_l^{\alpha}(\tau)}{1 + F_l^{\alpha}(\tau)\tau_l } 
\ee
We now use this last equation, together with the Girsanov's theorem (\ref{girseq1}) to get the following result.
Suppose to consider a $1$-dimensional Wiener process $Z^{(k, \alpha)}(\tau)$ where the $(k, \alpha)$-apex is used to stress that it is a Wiener process in the $t_k$-fwd measure of the $\alpha$ currency. We can then conclude that $Z^{(h, \beta)}(\tau)$ as defined in the following is a Wiener process in the $t_h$-fwd measure of the $\beta$ currency:
\be 
dZ^{(h, \beta)}(\tau) = dZ^{(k, \alpha)}(\tau)- s(h,k) \sum_{l=m_{h,k}+1}^{M_{h,k}} \frac{\tau_l  \left<d F_l^{\alpha}(\tau)\cdot dZ^{(k, \alpha)}(\tau)\right>}{1 + F_l^{\alpha}(\tau)\tau_l } -\\
- \sigma_{F_X(t_h)} \left<dW_X(\tau)\cdot dZ^{(k, \alpha)}(\tau)\right>
\label{tfwdmeasurechangedet}
\ee

%%%%%%%%%%%%%%%%%%%%%%%%%%%%%%%%%%%%%%%%%%%%%%%%%%%
% BIBLIOGRAFIA 																		%
%%%%%%%%%%%%%%%%%%%%%%%%%%%%%%%%%%%%%%%%%%%%%%%%%%%

\begin{thebibliography}{99}
\bibitem{Bjork} Bjork, 'Arbitrage Theory in Continuos Time', 2nd ed, Oxford\\


\end{thebibliography}



\end{document}
